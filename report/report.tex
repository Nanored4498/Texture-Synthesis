\documentclass[12pt]{article}

\usepackage[french]{babel}
\usepackage[utf8]{inputenc}  
\usepackage[T1]{fontenc}
\usepackage[left=2.5cm,right=2.5cm,top=3cm,bottom=2.65cm]{geometry}

\title{
	Computational Geometry and Digital Images\\
	\textbf{Texture Synthesis}\\
	Rapport
}
\author{
	Yoann Coudert--Osmont \\ \texttt{yoann.coudert-osmont@ens-lyon.fr}
	\and
	Jérémy Petithomme \\ \texttt{jeremy.petithomme@ens-lyon.fr}
}

\begin{document}
	
\maketitle

\section{Introduction}
	
La synthèse de texture consiste à créer une grande image à partir d'une petite image exemple ou à partir d'un modèle. La texture produite doit respecter une certaine structure mais si elle est produite à partir d'un exemple elle ne doit pas être une simple copie de l'exemple.

\end{document}